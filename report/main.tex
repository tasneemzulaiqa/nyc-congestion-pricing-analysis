% Setup - do not change
\documentclass[11pt]{article}
\usepackage[top=0.9in, left=0.9in, bottom=0.9in, right=0.9in]{geometry} 
\usepackage{parskip}

\usepackage[english]{babel}
\usepackage[utf8]{inputenc}
\usepackage{amsmath,amsthm,amssymb,graphicx,pdfpages,lipsum,hyperref}
\usepackage[none]{hyphenat}
\usepackage{csquotes}

\setlength\parindent{0pt}
%%%%%%%%%%%%%%%%%%%%%%%%%%%%%%%%%%%%%%%%%%%%%%%%%%%%%%%%%%%%%%%%%%%
% add other packages here if required
\usepackage{subcaption}
\usepackage[labelfont=bf]{caption}


%% Bibliography are specified in this file. You can also choose inline bib style if you want to. But make sure your citation style is consistent (and proper)
% For more details on citation: https://library.unimelb.edu.au/recite
\usepackage[sorting = none]{biblatex}
\usepackage{url}
\addbibresource{references.bib}
% Add additional packages here if required

%%%%%%%%%%%%%%%%%%%%%%%%%%%%%%%%%%%%%%%%%%%%%%%%%%%%%%%%%%%%%%%%%%% the '%' symbol denotes comments

% Begin document creation
% DELETE THE \lipsum PLACEHOLDERS WHEN YOU BEGIN
\title{\textbf{Impact of NYC Congestion Pricing on Yellow Taxi and HVFHV Demand}}
\author{
Tasneem Zulaiqa \\
Student ID: 1393971 \\
\href{https://github.com/MAST30034-AppliedDataScience/project-1-individual-tasneemzulaiqa/tree/main/code}{Github repo with commit}
}

\begin{document}
\maketitle

\section{Introduction}

New York City (NYC) experiences some of the worst traffic congestion in the world, with drivers losing an average of 102 hours per year due to delays\cite{TrafficScoreCard}. Even with a 24-hour subway system, Manhattan's central business district remains heavily congested. To help reduce congestion, the Metropolitan Transportation Authority (MTA) introduced the Central Business District Tolling Program (CBDTP) on January 5, 2025, which charges most vehicles entering Manhattan below 61st Street\cite{NYC311}. While the program aims to ease traffic, its early effects on mobility services are unclear, and it may put additional pressure on Yellow taxis which already face stiff competition from high-volume FHVs like Uber and Lyft.  

This study looks at how the CBDTP has affected Yellow taxi activity compared to high-volume FHVs, focusing on three main aspects: \textit{demand}, \textit{spatial patterns}, and \textit{traffic performance}. We also include hourly weather and geospatial factors to account for other influences on travel. The goal is to provide insights for drivers, operators, and policymakers about how taxi services are adapting and where potential challenges may arise.

\subsection{Dataset}

The primary data source for this study is the \textbf{TLC Trip Record Data} published by the NYC Taxi and Limousine Commission (TLC)\cite{TLCTripRecordData}. To assess the impact of congestion pricing, we focus on trips made by Yellow Taxis and High-Volume For-Hire Vehicles (HVFHVs) as they account for the majority of rides and thus provide the clearest picture of market-level changes. This focus is also motivated by their differing regulatory frameworks, which may cause them to respond differently to congestion pricing.

Our timeline covers January–June 2024 and January–June 2025. The year 2024 provides the baseline period before the introduction of CBDTP, while 2025 captures the immediate post-policy environment, which makes this a natural before–after comparison. For modeling purposes, we further restrict the analysis to January–March of both years to ensure consistent and complete weather records. This design ensures comparability while leveraging the most recent available data.  

To supplement the trip records, we incorporate hourly weather observations from the U.S. National Centers for Environmental Information (NCEI) \textbf{Global Historical Climatology Network-hourly (GHCNh)}\cite{NOAAGHCN}. For 2024, we use the Central Park Station data, while for 2025 we rely on JFK Airport station, as it provides broader coverage (Available until April). Weather data is included because temperature and precipitation are known to significantly affect travel demand and mode choice.  

Trip-level records are analyzed at multiple temporal and spatial resolutions to capture both general trends and detailed patterns in the use of mobility services. This allows us to assess the impact of CBDTP from different analytical perspectives.

The initial shapes of the dataset files are shown as below

\begin{table}[h!]
\centering
\begin{tabular}{lrr} % l=left, r=right, r=right
\hline
\textbf{Dataset} & \textbf{Instances} & \textbf{No. of Features} \\
\hline
TLC Yellow Taxi Trip Record Data 2024 & 20,332,093& 19\\
TLC Yellow Taxi Trip Record Data 2025 & 24,083,384& 20\\
TLC HVFHV Trip Record Data 2024 & 120,864,668& 24\\
TLC HVFHV Trip Record Data 2025 & 121,995,191& 25\\
NCEI GHCNh Dataset 2024& 5672& 234\\
NCEI GHCNh Dataset 2025& 3310& 234\\
\hline
\end{tabular}
\caption{Initial Datasets shape }
\end{table}

\section{Preprocessing}

The raw datasets described above are large in scale and contain various inconsistencies, which make preprocessing a critical step before conducting analysis and modeling. This section outlines the preprocessing steps applied to ensure data quality for subsequent analysis.

\subsection{Data Wrangling and Feature Selection}

\subsubsection{TLC Trip Record Data}

The TLC Trip Record Data were preprocessed using the following steps:

\begin{itemize}
    \item \textbf{Trip time:} For Yellow Taxi trips, duration (in minutes) was calculated from pickup and dropoff timestamps. For HVFHV trips, the provided trip time (in seconds) was converted to minutes for consistency.
    
    \item \textbf{Location mapping:} Pickup and dropoff location IDs were mapped to their corresponding zones using the TLC zone lookup table. Unused service zone columns were dropped.
    
    \item \textbf{Outlier filtering:}
    \begin{enumerate}
        \item [a.] Given that New York City's land area is approximately 300 square miles, any trip exceeding 200 miles is considered an outlier\cite{BoroughsOfNYC}.
        \item [b.] Trips with average speeds below 3 mph or above 80 mph were removed. These thresholds exclude trips that are either unreasonably slow or faster than highway limits. Average speed was computed from trip distance and duration.
        \item [c.] Trips with fares below \$3 (or voided trips) were removed, aligning with the Yellow Taxi base fare rule\cite{TLCTaxiFare}. The same threshold was applied to HVFHV driver pay.
        \item [d.] For Yellow Taxi trips only, records with fares inconsistent with the TLC fare formula (based on distance, duration, rate code, and speed) were excluded\cite{TLCTaxiFare}.
    \end{enumerate}
    
    \item \textbf{Zone validity:} Only trips with valid pickup and dropoff zones (per the TLC lookup file) were retained.
    
    \item \textbf{Missing values:} Rows with nulls in key fields were removed to ensure data completeness.
\end{itemize}


In total, the dataset was reduced from \textbf{287.3 million} raw trip records to \textbf{203.5 million} cleaned records after preprocessing. Specifically, Yellow Taxi trips decreased from \textbf{44.4 million} to \textbf{33.8 million} (24\% reduction), while HVFHV trips decreased from \textbf{242.9 million} to \textbf{169.7 million} (30\% reduction). These reductions were mainly done to ensure a reliable basis for analysis.

\subsubsection{NCEI GHCNh Dataset}

Overalll, most columns in the weather dataset were dropped as they contained metadata and quality flags for each variable. Other variables such as dew point or visibility had too many missing values to be useful. To keep the dataset consistent, only the main features \textbf{temperature, precipitation, and wind speed} were retained.

Records without temperature were removed since only two entries were missing. Missing precipitation values were filled with \textbf{0}, as it is reasonable to assume that no recorded precipitation means dry weather. For wind speed, missing values were imputed using the \textbf{median} as wind speed is a continuous variable and the median is more robust to extreme values than the mean.

\subsection{Feature Engineering and Data Aggregation}

Several additional features were derived to better capture fare dynamics, temporal patterns, and the effects of congestion pricing. Specifically, we engineered two fare-based metrics: \textbf{fare per mile} and \textbf{fare per minute}, computed by normalizing the total fare amount by trip distance and trip duration, respectively. These features allow for a consistent comparison of cost efficiency across different service types.

To identify trips affected by congestion pricing, a \textbf{CBD flag} was created. For 2024, prior to the policy, this was based on whether the pick-up or drop-off location fell within designated CBD zones. For 2025,  the flag was derived directly from the \verb|cbd_congestion_fee| field.

Temporal features were also extracted such as \textbf{{day of week}}, a \textbf{weekend indicator}, and a \textbf{peak-hour flag} marking trips during typical commuting periods (6-10 AM and 4-8 PM). These variables capture temporal patterns in demand.

After feature engineering, trip records were aggregated at multiple levels to support different analyses and modelling approaches. Aggregations were chosen to align with the analysis type: yearly summaries for overall comparisons, monthly and daily summaries to support regression models capturing temporal variation. These aggregated datasets formed the basis of exploratory data analysis and subsequent modelling.

\section{Preliminary Analysis}

\subsection{Distribution of Pick Up Demand}

The distribution of trips across NYC in 2024 was examined to understand baseline demand patterns for Yellow Taxis and HVFHVs. Figure~\ref{fig:image1}  shows total trips aggregated by pickup zones for both services. For Yellow Taxis, the CBD has the highest concentration of trips, with airports such as JFK and LaGuardia also accounting for a substantial share. The top three zones are Midtown Center, Upper East South Side, and JFK Airport. In contrast,
HVFHV trips are more spatially dispersed accross both the CBD and the outer boroughs, with the top zones being LaGuardia Airport, JFK Airport, and the Crown Heights North.


\begin{figure}[h]

\begin{subfigure}{0.5\textwidth}
\includegraphics[width=0.8\linewidth, height=7cm]{yellow_demand_24.png} 
\caption{Yellow Taxi 2024}
\label{fig:subim1}
\end{subfigure}
\begin{subfigure}{0.5\textwidth}
\includegraphics[width=0.8\linewidth, height=7cm]{hvfhv_demand_24.png}
\caption{HVFHV 2024}
\label{fig:subim2}
\end{subfigure}

\caption{Pick-up demand distribution for Yellow Taxi vs HVFHV in 2024}
\label{fig:image1}
\end{figure}

Since the data was skewed, an outlier check using the IQR method on log trip count confirmed no extreme anomalies for Yellow Taxis, while a few very low-demand HVFHV zones were identified. These zones were retained in the analysis as they reflect expected service patterns.

\subsection{Distribution of Market Share}

To assess competitiveness more fairly, we examined market share by zone, which reflects the relative demand of each service in each area. 

\begin{figure}[h]

\begin{subfigure}{0.5\textwidth}
\includegraphics[width=0.8\linewidth, height=7cm]{yellow_market_share_24.png} 
\caption{Yellow Taxi 2024}
\label{fig:subim1}
\end{subfigure}
\begin{subfigure}{0.5\textwidth}
\includegraphics[width=0.8\linewidth, height=7cm]{hvfhv_market_share_24.png}
\caption{HVFHV 2024}
\label{fig:subim2}
\end{subfigure}

\caption{Market Share distribution for Yellow Taxi vs HVFHV in 2024}
\label{fig:image2}
\end{figure}

For Yellow Taxis, market share is generally low across most boroughs, with higher shares at Newark Airport, Central Park, and Rikers Island. The CBD, previously dominated in raw trip counts, now shows only moderate market share.

HVFHVs display the opposite pattern, holding strong market share citywide, with lower relative share in Newark, Rikers Island, and the CBD. Notably, the top three HVFHV pickup zones are located in more distant areas such as Staten Island. This suggests that HVFHVs may be preferred for longer or less centrally located trips, where yellow taxis are less competitive or less readily available.

At major airports such as JFK and LaGuardia, both services show more balanced market share, suggesting a more balanced market share between Yellow Taxis and HVFHVs in these locations. Overall, this analysis highlights areas where each service dominates and zones where the market is more evenly split.

\subsection{Comparison Across Years (2024 vs 2025)}

Using the aggregated CBD and non-CBD trip dataset, we compared market shares across years to assess shifts in demand and potential congestion effects. 

For CBD trips, yellow taxis increased their share from \textbf{28.9\% in 2024} to \textbf{31.2\% in 2025}, while HVFHV services declined slightly from \textbf{71.1\%} to \textbf{68.8\%}. This indicates a modest recovery for yellow taxis within the CBD. In contrast, non-CBD trips showed minimal change, with yellow taxis maintaining a consistently small share (approximately \textbf{7\%}) across both years, and HVFHV services continuing to dominate.

\begin{figure}[h]
    \centering
    \includegraphics[width=0.4\textwidth]{market_share_change.png}
    \caption{Percentage-Point Changes in Market Share} 
    \label{fig:image3}
\end{figure}

Figure~\ref{fig:image3} highlights these changes. One potential driver of this shift is the Congestion Relief Zone toll structure: yellow taxis are charged only \$0.75 per trip within the zone, compared to \$1.50 for HVFHV\cite{NYC311}. This lower per-trip cost may make yellow taxis more attractive for CBD trips, especially when combined with improved traffic conditions.

Next, to examine the role of congestion, we analyzed average speeds for trips fully within the CBD. Speeds were filtered to align with local limits so that the analysis reflects typical traffic conditions rather than outliers or trips entering and leaving the area\cite{NYC_SpeedLimits}.

\begin{figure}[h]
    \centering
    \includegraphics[width=0.4\textwidth]{congestion_relief.png}
    \caption{Changes in Average Speeds within CBD} 
    \label{fig:image4}
\end{figure}

Figure~\ref{fig:image4} shows 35 CBD zones experienced improved speeds in 2025 compared to 2024, while 3 zones worsened. This suggests that the congestion relief program is effective at making trips through the CBD faster. Combined with lower yellow taxi fares relative to HVFHV services and high private vehicle tolls, this likely increased the attractiveness of yellow taxis, which provides a plausible explanation for the observed market share gains within the CBD.

\section{Modelling}

\subsection{Ordinary Least Squares (OLS) Regression}

An OLS regression was applied to the monthly aggregated data to obtain interpretable parameter estimates. The dependent variable was the logit-transformed market share as the market share is a proportion bounded between 0 and 1. Explanatory variables included month, service type, year, and CBD status, all treated as categorical and encoded with dummy variables. Interaction terms between year, service type, and CBD status were added to capture changes in competitiveness over time and across locations. Continuous covariates such as average speed, fares, weather, and congestion fees were included to account for variation in travel conditions and pricing. The model assumes linear relationships between the predictors and the log-odds of market share.

\subsection{Random Forest Regression}

Random Forest Regression (RFR) was chosen for its robustness and its ability to capture non-linear relationships and higher-order interactions. The model was trained on daily aggregated data to predict market share at a finer temporal scale and capture short-term demand dynamics. Similar predictors were used, but unlike OLS, RFR does not require interaction terms to be specified and also provides feature importance measures.

\section{Results and Discussion}

The OLS regression explained a substantial proportion of the variation in market share ($R^{2}=0.693$). Table~\ref{tab:ols_key_coeff} summarises the key coefficients.

\begin{table}[ht]
\centering
\begin{tabular}{lrrrr}
\hline
\textbf{Variable} & \textbf{Coefficient} & \textbf{Std. Error} & \textbf{p-value} \\
\hline
Intercept & -5.9833 & 0.119 & 0.000 \\
C(service\_type)[T.yellow] & -5.5033 & 0.103  & 0.000 \\
year\_2025 & -0.3288 & 0.143  & 0.021 \\
year\_2025:C(service\_type)[T.yellow] & 0.3684 & 0.138  & 0.008 \\
is\_cbd & -1.2559 & 0.152 & 0.000 \\
C(service\_type)[T.yellow]:is\_cbd & 2.0690 & 0.145 & 0.000 \\
avg\_speed & -0.8738 & 0.028  & 0.000 \\
avg\_time & -1.3201 & 0.034 & 0.000 \\
avg\_precip\ & 0.1277 &  0.029 & 0.000 \\
\hline
\end{tabular}
\caption{Key OLS Regression Coefficients for Market Share Model}
\label{tab:ols_key_coeff}
\end{table}

The results suggest that yellow taxis had a lower baseline market share than HVFHV services outside the CBD in 2024 but partially regained market share in 2025, which indicates some recovery following policy changes. Within the CBD, Yellow taxis performed relatively better than HVFHV services. Longer or slower trips were associated with reduced market share, whereas rainy conditions slightly increased demand for Yellow taxis. While multicollinearity among predictors may affect coefficient precision, the significant interaction effects indicate these patterns are robust for interpreting relative competitiveness.

\begin{figure}[h]
    \centering
    \begin{minipage}{0.45\textwidth}
        \centering
        \label{tab:rmse}
        \begin{tabular}{l c}
        \hline
        \textbf{Dataset} & \textbf{RMSE} \\
        \hline
        Train & 0.0428 \\
        Validation & 0.0519 \\
        Test & 0.1712 \\
        \hline
        \end{tabular}
        \captionof{table}{RMSE on Datasets}
     \end{minipage}\hspace{0.1 cm} 
    \begin{minipage}{0.5\textwidth}
        \centering
        \includegraphics[width=\textwidth]{feat_imp.png}
        \caption{Top Feature Importance}
        \label{fig:feat_imp}
    \end{minipage}
\end{figure}

For the RFR model, the RMSE on the training and validation sets shows no major signs of overfitting. The higher test RMSE is expected, as the test data covers the after policy changes that altered market dynamics and daily aggregation introduces additional variability. Rather than indicating poor model performance, this highlights that the test set captures new patterns. The focus is therefore on whether the model correctly identifies the key drivers of market share, rather than on the exact RMSE values.

Figure~\ref{fig:feat_imp} shows that congestion surcharges and service type were the most influential predictors, followed by fare per mile and trip duration, while pickup zones had smaller contributions. Zones with higher average congestion surcharges are usually busier areas. In these zones, yellow taxi market share tends to be lower compared to HVFHVs, as people may prefer app-based rides or walk in congested areas. Longer trips and higher fares per mile also favor HVFHVs, which are easier to book and generally more convenient for longer journeys. 

Taken together, both models suggest a consistent narrative. Yellow taxis retain an advantage in the Manhattan core but face structural disadvantages on faster, longer trips dominated by HVFHVs. The RFR model strengthens this conclusion by illustrating how market share varies with congestion, service type, fare per mile, and trip duration, providing actionable insights for service planning.

\section{Recommendations}

Based on our analysis, we provide two key recommendations for drivers, mobility operators, and policymakers. First, \textbf{focus service in dense and high-demand areas}. Both OLS and RFR models show that Yellow taxis maintain a relative advantage in Manhattan’s core, particularly in high-demand zones such as Midtown and Upper East South Side. Drivers and operators should prioritize availability in these areas to capture the highest market share, while policymakers can ensure adequate coverage to support urban accessibility and maintain competitive balance with HVFHVs. Second, \textbf{adapt to trip characteristics and fare patterns}. Analysis indicates that longer trips and higher fares per mile favor HVFHVs, and higher congestion reduces yellow taxi market share. Drivers can strategically target shorter trips in moderate-traffic areas where yellow taxis are more competitive. Moreover, operators can adjust incentives and routing to match these patterns, and policymakers can consider measures that maintain fairness in pricing and accessibility across service types.

\section{Conclusion}

This analysis looked at yellow taxi market share compared to HVFHV in NYC using OLS and RFR models. Yellow taxis still do well in busy areas like central Manhattan, while longer trips, higher fares per mile, and congested areas tend to favor HVFHVs. However, the study has some limitations. It uses aggregated data which can hide short-term changes and individual trip behavior. Additionally, features like average congestion surcharges only roughly reflect real traffic. Future work could include more detailed data such as street blockages or accidents, and more sophisticated models to capture complex patterns. Nonetheless, the findings give useful insights for drivers, mobility operators, and policymakers trying to understand market trends

\clearpage

% BEGIN REFERENCES SECTION
\printbibliography

\end{document}